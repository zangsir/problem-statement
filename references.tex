E. Chang et al., �Large vocabulary Mandarin speech recognition with different approaches in modeling
tones,� in Proc. ICSLP, 2000, vol. 2, pp. 983�986.

Patel, A. D., Xu,Y. and Wang, B.: The role of F0 variation in the intelligibility of Mandarin sentences. In Proceedings of Speech Prosody 2010, Chicago.(2010).

Yi Xu, �Transimiting tone and intonation simultaneoulsy:the parallel encoding and target approximation (PENTA) model,� International Symposium on Tonal Aspects of Languages: With Emphasis on Tone Languages,pp. 215�220, 2004.

Greg Kochanski and Chilin Shih, �Prosody modeling with soft templates,� Speech Communication,vol. 39, no. 3-4, pp. 311�352, 2003.

Xuejing Sun, The determination, analysis, and synthesis of fundamental frequency,Ph.D. thesis, Northwester University, 2002.

Jinsong Zhang and Keikichi Hirose, �Tone nucleus modeling for Chinese lexical tone recognition,� Speech Communication,vol. 42, pp. 447�466, 2005.

Gina-Anne Levow, �Context in multi-lingual tone and pitch accent recognition,� International Conference on Speech Communication and Technology,2005.

Chao Wang and Stephanie Seneff, �Improving tone recognition by normalizing for coarticulation and intonation effects,� International Conference on Spoken Language Processing,2000.

Yi Xu and D.H Whalen, �Information for mandarin tones in the amplitude contour and in brief segments,� Phonetica, vol. 49, pp. 25�47, 1992

I. Fischer and J. Poland, �New methods for spectral clustering,� Tech. Rep. ISDIA-12-04, IDSIA, 2004.

Mikhail Belkin and Partha Niyogi, �Laplacian eigenmaps and spectral techniques for embedding and clustering,� in Proceeding of NIPS�02, 2002.

C.X.Xu, Y. Xu, and L.-S. Luo, �A pitch target approximation model for f0 contours in Mandarin,� in Proceedings of the 14th International Congress of Phonetic Sciences, 1999, pp. 2359�2362.

Yi Xu and X. Sun, �Maximum speed of pitch change and how it may relate to speech,� Journal of the Acoustical Society of America, vol. 111, 2002.

J. Lin, E. Keogh, S. Lonardi, and B. Chiu, �A symbolic representation of time series, with implications for streaming algorithms,� in Proc. of the 8th ACM SIGMOD workshop on Research issues in data mining and knowledge discovery, New York, USA, 2003, pp. 2�11.

A. Mueen, E. Keogh, Q. Zhu, S. Cash, and B. Westover, �Exact discovery of time series motifs,� in Proc. of SIAM Int. Con. on Data Mining (SDM), 2009, pp. 1�12.

T. Rakthanmanon, B. Campana, A. Mueen, G. Batista, B. Westover, Q. Zhu, J. Zakaria, and E. Keogh, �Addressing big data time series: mining trillions of time series subsequences under dynamic time warping,� ACM Transactions on Knowledge Discovery from Data (TKDD), vol. 7, no. 3, pp. 10:1�10:31, Sep. 2013. 

S. Gulati, J. Serr�, V. Ishwar and X. Serra, "Mining Melodic Patterns in Large Audio Collections of Indian Art Music." in Proceedings of International Conference on Signal Image Technology \& Internet Based Systems (SITIS) - Multimedia Information Retrieval and Applications, Marrakech, Morocco 2014.

C. Faloutsos, M. Ranganathan, and Y. Manolopulos. Fast Subsequence Matching in Time-Series Databases. SIGMOD Record. vol. 23. pp. 419-429. 1994.

Zhang, S. Analyze linguistic tone patterns using time-series mining techniques. Workshop on Computational Phonology and Morphology (CompMorPhon15), Linguistic Summer Institute (Big Data), University of Chicago, July 11, 2015a.

Zhang, S, Caro, R, Serra,X,. Study of the similarity between linguistic tones and melodic pitch contours in Beijing Opera singing. Proceedings of The 15th International Society for Music Information Retrieval (ISMIR) Conference, pp.345-348. Taiwan, October, 27-31 2014. 

Zhang,S, Caro, R, Serra,X. Predicting pairwise pitch contour relations based on linguistic tone information in Beijing opera singing. Proceedings of the 16th International Society for Music Information Retrieval (ISMIR) conference, Malaga, Spain, October 26th-30th, 2015b.

Keogh, E. (2002). Exact indexing of dynamic time warping. In 28th International Conference on Very Large Data Bases. Hong Kong. pp 406-417. 

S. Gulati, J. Serr� and X. Serra, "An Evaluation of Methodologies for Melodic Similarity in Audio Recordings of Indian Art Music", in Proceedings of IEEE Int. Conf. on Acoustics, Speech, and Signal Processing (ICASSP) (In Press), Brisbane, Australia 2015.

Xu, Y., Lee, A., Prom-on, S. \& Liu, F. (in press). Explaining the PENTA model: A reply to Arvaniti and Ladd (2009). Phonology (in press).

Xu, Y. and Prom-on, S. (2014). Toward invariant functional representations of variable surface fundamental frequency contours: Synthesizing speech melody via model-based stochastic learning. Speech Communication 57, 181-208.

Prom-on, S., Xu, Y. and Thipakorn, B. (2009). Modeling tone and intonation in Mandarin and English as a process of target approximation. Journal of the Acoustical Society of America 125: 405-424.

Gauthier, B., Shi, R. and Xu, Y. (2007). Learning phonetic categories by tracking movements. Cognition 103: 80-106. 

Xu, Y. (1997). Contextual tonal variations in Mandarin. Journal of Phonetics 25: 61-83. 

Xu, Y. (1994). Production and perception of coarticulated tones. Journal of the Acoustical Society of America 95: 2240-2253.

Can Voice Quality help Mandarin Tone Recognition? by Dinoj Surendran and Gina-Anne Levow, Proceedings of ICASSP 2008. 

Tone Recognition in Mandarin using Focus. Dinoj Surendran, Gina-Anne Levow, Yi Xu (Phonetics Department, UCL). Proceedings of the 9th European Conference of Speech Communication and Technology (Interspeech/ICSLP 2005) 

The functional load of tone in Mandarin is as high as that of vowels. Dinoj Surendran and Gina-Anne Levow. Proceedings of Speech Prosody 2004, Nara, Japan, pp. 99-102.

Xu, Y., Xu, C. X., Sun. X. �On the temporal domain of focus�, Proc. Intl. Conf. Speech Prosody, Nara, Japan. 1:81�94, 2004.

Xu, Y. �Effects of tone and focus on the formation and alignment of f0 contours�, J. Phonetics 27:55�105, 1999.

"Modeling Broad Context for Tone Recognition with Conditional Random Fields", Siwei Wang and Gina-Anne Levow, in Proceedings of Interspeech 2011, 2011.

"Improving Tone Recognition with Combined Frequency and Amplitude Modelling", Siwei Wang and Gina-Anne Levow, Proceedings of Interspeech 2006, p. 2386-2389.

Unsupervised and Semi-supervised Learning of Tone and Pitch Accent ", Gina-Anne Levow, HLT-NAACL 2006, p. 224-231. 

Unsupervised learning of tone and pitch accent. Gina-Anne Levow, Speech Prosody 2006, Dresden, Germany, May 2006. 